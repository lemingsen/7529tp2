\documentclass[../tp2_grupo404.tex]{subfiles}

\graphicspath{{\subfix{../out/}}}

\begin{document}

\begin{table}[H]
    \centering
    \begin{tabular}{@{}lcc@{}}
    \toprule
    \multicolumn{1}{c}{\textbf{Algoritmo}}                 & \textbf{Temporal}          & \textbf{Espacial} \\ \midrule
    \hyperref[sec:complej_Bellman]{\textbf{Bellman-Ford}}  & $O(V^2\times E)$           & $O(V+E)$     \\
    \hyperref[sec:complej_Floyd]{\textbf{Floyd-Warshall}}  & $\Theta(V^3)$              & $O(V^2)$          \\
    \hyperref[sec:complej_Johnson]{\textbf{Johnson}}       & $O(V^2 lg(V) + V\times E)$ & $O(V^2+V+E)$      \\ \bottomrule
    \end{tabular}
    \caption{\label{tabComplejidades}Complejidades. $V$=vértices, $E$=aristas.}
\end{table}

\subsubsection{Algoritmo de Bellman-Ford}\label{sec:complej_Bellman}
La complejidad temporal es $O(V\times E)$ por vértice de origen;
sin embargo a los fines de comparación se debe tomar en cuenta
la complejidad para analizar todo el grafo (cada vértice como origen),
lo que lo convierte en $O(V^2\times E)$.

\subsubsection{Algoritmo de Floyd-Warshall}\label{sec:complej_Floyd}

\subsubsection{Algoritmo de Johnson}\label{sec:complej_Johnson}

% FIN DEL DOCUMENTO (SECCIÓN P1.2)
% NO BORRAR POR ACCIDENTE NI ESCRIBIR COSSA ABAJO
\end{document}
