\documentclass[../tp2_grupo404.tex]{subfiles}

\graphicspath{{\subfix{../out/}}}

\begin{document}

\subsubsection{Convertir a grafo sin negativos (Bellman-Ford)}\label{sec:parte1_4_1}
Para el ejemplo tomaremos un grafo G arbitrario (\cref{grafoG}).
Como tiene pesos negativos, a partir de éste, creamos un grafo $G^\prime$ (\cref{grafoG_prima})
en principio idéntico pero al que le agregamos temporalmente un
vértice $V_0$ con aristas de peso $0$ hacia cada otro nodo preexistente.

\begin{figure}[H]
    \centering
    \subcaptionbox
        {\label{grafoG}\textbf{Grafo $G$ (original)}}
        {\includegraphics[width=0.4\linewidth,angle=0,origin=c]{out/ford/ford1A.png}}
    \subcaptionbox
        {\label{grafoG_prima}\textbf{Grafo $G^\prime$ (con nodo temporal 0)}}
        {\includegraphics[width=0.4\linewidth,angle=0,origin=c]{out/ford/ford1B.png}}
\end{figure}

A partir de este grafo, aplicamos el algoritmo de Bellman-Ford tomando
como origen el vértice temporal $V_0$. Para el mismo, en cada iteración,
recorreremos todas las aristas haciendo una reducción;
en términos generales:
\begin{displayquote}
Siendo $h(x)$ la distancia al origen estimada para el vértice $x$,
y $w(u,v)$ el peso original de la arista que va desde el vértice $u$ hasta $v$,
comparamos $h(u)+w(u,v)$ contra $h(v$). Si el primer valor es inferior al
segundo, actualizamos $h(v) := h(u)+w(u,v)$ y anotamos $u$ como padre de $v$:
$\pi(v):=u$; esto es, qué otro vértice se tomó para calcular $h(v)$.
\end{displayquote}

Iteramos estas reducciones sobre todas las aristas, un total de
$\lvert V \rvert-1$ veces, y luego una vez más para comprobar que
no haya algún ciclo negativo (que no haya cambios). Adicionalmente,
es posible finalizar con un resultado correcto (es decir, no hay ciclos)
cuando durante una iteración externa no se produjo cambio alguno.
\footnote{Si no hay cambios en la iteración, los datos de entrada de
la siguiente iteración serán los mismos; por lo que el resultado también
será el mismo (y seguirá sin haber cambios). Si bien no modifica
big O, puede reducir el tiempo real de cómputo.}

\begin{table}[H]
    \centering
    \caption{\centering Primera iteración de Bellman-Ford en el algoritmo de Johnson}
    \begin{tabular}{@{}cccccccccccccc@{}}
    \toprule
     & \multicolumn{3}{c}{1ª iteración}   & \multicolumn{2}{c}{$V_1$} & \multicolumn{2}{c}{$V_2$}  & \multicolumn{2}{c}{$V_3$} & \multicolumn{2}{c}{$V_4$} & \multicolumn{2}{c}{$V_5$} \\
     \multirow{-2}{*}{\begin{tabular}[c]{@{}c@{}}Fórmula\\ posible nuevo\end{tabular}}&Nuevo&$\lesseqgtr$& Ant. & h &    p   & h                           & p                             & h                           & p                                    & h & p      & h & p    \\ \midrule
    $0\rightarrow 1 \dots 0\rightarrow 5$   & {\color[HTML]{9A0000} $0+0$} & $<$ & 0                           & 0 & $V_0$ & 0                           & $V_0$                        & 0                           & $V_0$                               & 0 & $V_0$ & 0 & $V_0$ \\
    $h(1)+w(1,2)$ & $0+12$                          & $<$                     & 0                           & 0 & $V_0$ & 0                           & $V_0$                        & 0                           & $V_0$                                  & 0 & $V_0$ & 0 & $V_0$ \\
    $h(1)+w(1,3)$ & {\color[HTML]{9A0000} $0+(-3)$} & {\color[HTML]{9A0000} $<$} & {\color[HTML]{9A0000} $0$}  & 0 & $V_0$ & 0                           & $V_0$                        & {\color[HTML]{9A0000} $-3$} & {\color[HTML]{9A0000} $V_1$}        & 0 & $V_0$ & 0 & $V_0$ \\
    $h(2)+w(2,3)$ & $0+-1$                          & $>$                        & -3                          & 0 & $V_0$ & 0                           & $V_0$                        & -3                          & $V_1$                               & 0 & $V_0$ & 0 & $V_0$ \\
    $h(2)+w(2,4)$ & $0+15$                          & $>$                        & 0                           & 0 & $V_0$ & 0                           & $V_0$                        & -3                          & $V_1$                               & 0 & $V_0$ & 0 & $V_0$ \\
    $h(3)+w(3,5)$ & $-3+21$                         & $>$                        & 0                           & 0 & $V_0$ & 0                           & $V_0$                        & -3                          & $V_1$                               & 0 & $V_0$ & 0 & $V_0$ \\
    $h(4)+w(4,2)$ & {\color[HTML]{9A0000} $0+(-2)$} & {\color[HTML]{9A0000} $<$} & {\color[HTML]{9A0000} $0$}  & 0 & $V_0$ & {\color[HTML]{9A0000} $-2$} & {\color[HTML]{9A0000} $V_4$} & -3                          & $V_1$                               & 0 & $V_0$ & 0 & $V_0$ \\
    $h(5)+w(5,1)$ & $0+8$                           & $>$                        & 0                           & 0 & $V_0$ & -2                          & $V_4$                        & -3                          & $V_1$                               & 0 & $V_0$ & 0 & $V_0$ \\
    $h(5)+w(5,3)$ & {\color[HTML]{9A0000} $0+(-4)$} & {\color[HTML]{9A0000} $<$} & {\color[HTML]{9A0000} $-3$} & 0 & $V_0$ & -2                          & $V_4$                        & {\color[HTML]{9A0000} $-4$} & {\color[HTML]{9A0000} $V_5$}        & 0 & $V_0$ & 0 & $V_0$ \\
    $h(5)+w(5,4)$ & $0+13$                          & $>$                        & 0                           & 0 & $V_0$ & -2                          & $V_4$                        & -4                          & $V_5$                               & 0 & $V_0$ & 0 & $V_0$ \\ \bottomrule
    \end{tabular}
    \label{EjFord1}
\end{table}

\begin{table}[H]
    \centering
    \caption{\centering Segunda iteración de Bellman-Ford en el algoritmo de Johnson}
    \begin{tabular}{@{}cccccccccccccccccc@{}}
    \toprule
     & \multicolumn{3}{c}{2ª iteración} & \\
     \multirow{-2}{*}{\begin{tabular}[c]{@{}c@{}}Fórmula\\ posible nuevo\end{tabular}}&Nuevo&$\lesseqgtr$& Ant. & \multirow{-2}{*}{Resultados} \\ \cmidrule(r){1-5}
    $0\rightarrow 1 \dots 0\rightarrow 5$   &  $0+0$     & $\geq$   & h(x) & Sin cambios                  \\
    $h(1)+w(1,2)$ & $0+12$    & $>$ & $-2$   & Sin cambios                  \\
    $h(1)+w(1,3)$ & $0+(-3)$  & $>$ & $-4$   & Sin cambios                  \\
    $h(2)+w(2,3)$ & $-2+(-1)$ & $>$ & $-4$   & Sin cambios                  \\
    $h(2)+w(2,4)$ & $-2+15$   & $>$ & $0$    & Sin cambios                  \\
    $h(3)+w(3,5)$ & $-4+21$   & $>$ & $0$    & Sin cambios                  \\
    $h(4)+w(4,2)$ & $0+(-2)$  & $>$ & $-2$   & Sin cambios                  \\
    $h(5)+w(5,1)$ & $0+8 $    & $>$ & $0$    & Sin cambios                  \\
    $h(5)+w(5,3)$ & $0+(-4)$  & $>$ & $-4$   & Sin cambios                  \\
    $h(5)+w(5,4)$ & $0+13$    & $>$ & $0$    & Sin cambios                  \\ \bottomrule
    \end{tabular}
    \label{EjFord2}
\end{table}

Como puede observarse en el \Cref{EjFord1}, en las primeras
reducciones simplemente se actualiza la distancia
estimada hasta cada vértices como \texttt{0};
esto siempre es así, por definición.
\footnote{Como se agregó una arista de peso 0 desde
el origen hasta cada nodo, y las mejores distancias se inicializan
en infinito excepto el origen mismo.}
Por esto lo hemos marcado en una sola fila (véase \cref{fig:FordCambios0}).
En las siguientes reducciones, sólo hay tres en las que se realiza una actualización:
\begin{enumerate}
    \item[$V_1\overset{-3}{\rightarrow} V_3$] (\cref{fig:FordCambios1}) El vértice $V_3$
        tenía una distancia estimada de $0$, que es actualizada a $-3$
        ( que resulta de la suma de la distancia estimada de $h(V_1)=0$
        y el peso original de la arista $w(1,3)=-3$).
    \item[$V_4\overset{-2}{\rightarrow} V_2$] (\cref{fig:FordCambios2}) La distancia estimada al vértice $V_2$
        era $0$, y es actualizada a $-2$ (por la suma de la $h(V_4)$ y el peso
        original de la arista $w(4,2)=-2$).
    \item[$V_5\overset{-4}{\rightarrow} V_3$] (\cref{fig:FordCambios3}) Nos encontramos con otra arista que
        llega al nodo $V_3$. Su distancia estimada ahora se actualiza a $-4$,
        ya que $h(5)+w(5,4)=0+(-4)=-4$ es menor a la distancia estimada la momento ($-3$).
\end{enumerate}

\begin{figure}[H]
    \centering
    \subcaptionbox
        {\label{fig:FordCambios0}\textbf{Estimación inicial}}
        {\includegraphics[width=0.4\linewidth,angle=0,origin=c]{out/ford/ford1C.png}}
    \subcaptionbox
        {\label{fig:FordCambios1}\textbf{Estimación para vértice 3 es -3.}}
        {\includegraphics[width=0.4\linewidth,angle=0,origin=c]{out/ford/ford1D.png}}
%\end{figure}
%\begin{figure}[H]
    \centering
    \subcaptionbox
        {\label{fig:FordCambios2}\textbf{Estimación para vértice 2 es -2.}}
        {\includegraphics[width=0.4\linewidth,angle=0,origin=c]{out/ford/ford1E.png}}
    \subcaptionbox
        {\label{fig:FordCambios3}\textbf{Vuelve a cambiar la estimación para vértice 3.}}
        {\includegraphics[width=0.4\linewidth,angle=0,origin=c]{out/ford/ford1F.png}}
        \caption{\label{fig:EjFord1Cambios}\textbf{Grafo con cambios mediante Bellman-Ford}.}
\end{figure}

Mientras que en la segunda iteración, si bien los valores de $h(x)$
son distintos, el resultado de la comparación es el mismo
por lo que no hay ningún cambio en ésta ni las siguientesa,
finalizando exitosamente el algoritmo de Bellman-Ford.
En este punto, se crea un nuevo grafo $H$ posteriormente empleado
para calcular los caminos mediante el algoritmo de Dijkstra.
Se toma como base el grafo $G$, actualizando los pesos de las aristas
según la fórmula:
$$\hat{w}(u,v)~:=~w(u,v) + h(u) - h(v)$$
Donde:
\begin{itemize}
    \item[$w(u,v)$] es el peso original de la arista, en el grafo $G$, desde el vértice $u$ hasta el vértice $v$.
    \item[$h(x)$] es la distancia estimada, en el grafo $G^\prime$, desde $V_0$ hasta $x$.
    \item[$\hat{w}$] es el nuevo peso, en el grafo $H$, desde el vértice $u$ hasta el vértice $v$.
\end{itemize} 

\begin{figure}[H]
    \centering
    \subcaptionbox
        {\label{fig:GrafoH_calculos}\textbf{Cálculo de cada arista para sus pesos}.}
        {\includegraphics[width=0.4\linewidth,angle=0,origin=c]{out/ford/ford1G.png}}
    \subcaptionbox
        {\label{fig:GrafoH_valores}\textbf{Grafo H con los valores calculados}.}
        {\includegraphics[width=0.4\linewidth,angle=0,origin=c]{out/ford/ford1H.png}}
    \caption{\label{fig:GrafoH}\textbf{Grafo $H$} con cambios de pesos con $\hat{w}$.}
\end{figure}

\subsubsection{Aplicar Dijkstra}\label{sec:parte1_4_2}

\begin{table}[H]
\begin{subtable}[H]{0.45\textwidth}
    \centering
    \begin{tabular}{@{}lccccc@{}}
    \toprule
    \textbf{origen} & \textbf{V\textsubscript{1}} & \textbf{V\textsubscript{2}} & \textbf{V\textsubscript{3}} & \textbf{V\textsubscript{4}} & \textbf{V\textsubscript{5}} \\ \midrule
    \textbf{V\textsubscript{1}}  & 0   & 14  & 1  & 27  & 18  \\
    \textbf{V\textsubscript{2}}  & 26  & 0   & 1  & 13  & 18  \\
    \textbf{V\textsubscript{3}}  & 25  & 30  & 0  & 30  & 17  \\
    \textbf{V\textsubscript{4}}  & 26  & 0   & 1  & 0   & 18  \\
    \textbf{V\textsubscript{5}}  & 8   & 13  & 0  & 13  & 0   \\ \bottomrule
    \end{tabular}
    \caption{Distancia mínima}
    \label{DijkstraGrafoHMinimos}
\end{subtable}
\hfill
\begin{subtable}[H]{0.45\textwidth}
    \centering
    \begin{tabular}{@{}lccccc@{}}
    \toprule
    \textbf{origen} & \textbf{V\textsubscript{1}} & \textbf{V\textsubscript{2}} & \textbf{V\textsubscript{3}} & \textbf{V\textsubscript{4}} & \textbf{V\textsubscript{5}} \\ \midrule
    \textbf{V\textsubscript{1}}  & -                   & V\textsubscript{1}  & V\textsubscript{1}  & V\textsubscript{2}  & V\textsubscript{3}  \\
    \textbf{V\textsubscript{2}}  & V\textsubscript{5}  & -                   & V\textsubscript{2}  & V\textsubscript{2}  & V\textsubscript{3}  \\
    \textbf{V\textsubscript{3}}  & V\textsubscript{5}  & V\textsubscript{4}  & -                   & V\textsubscript{5}  & V\textsubscript{3}  \\
    \textbf{V\textsubscript{4}}  & V\textsubscript{5}  & V\textsubscript{4}  & V\textsubscript{2}  & -                   & V\textsubscript{3}  \\
    \textbf{V\textsubscript{5}}  & V\textsubscript{5}  & V\textsubscript{4}  & V\textsubscript{5}  & V\textsubscript{5}  & -                   \\ \bottomrule
    \end{tabular}
    \caption{Nodo antecesor a cada destino}
    \label{DijkstraGrafoHAntecesor}
    \end{subtable}
\caption{Resultados al aplicar Dijkstra}
\label{DijkstraGrafoH}
\end{table}

\subsubsection{Restaurar valor de distancias}\label{sec:parte1_4_3}
Finalmente, es importante tener en cuenta que el resultado de Dijkstra
nos muestra los caminos mínimos, pero con las distancias alteradas $\hat{\delta}(u,v)$
(basadas en los pesos $\hat{w}$). Entonces un último paso, para cada
iteración de Dijkstra (por cada vértice de origen), es restaurar la
verdadera distancia hasta cada vértice destino $v$:
$$d(u,v) ~:=~ \hat{w}(u,v) + h(v) - h(u)$$

Tomando entonces $h(x)$ que calculamos de Bellman-Ford,
actualizamos (\cref{AjusteJohnsonCalculos}) cada distancia mínima de \cref{DijkstraGrafoHMinimos}
para obtener la tabla \cref{AjusteJohnsonValores} de distancias mínimas reales.

\begin{table}[H]
    \centering
    \begin{tabular}{@{}lccccc@{}}
    \toprule
    \textbf{origen}
    & \textbf{h(V\textsubscript{1})= 0}
    & \textbf{h(V\textsubscript{2})=-2}
    & \textbf{h(V\textsubscript{3})=-4}
    & \textbf{h(V\textsubscript{4})= 0}
    & \textbf{h(V\textsubscript{5})= 0} \\ \midrule
    \textbf{h(V\textsubscript{1})= 0}  & 0 +0-0  & 14+(-2)-0  & 1+(-4)-0  & $27~\cancel{+0-0}$  & $18\cancel{+0-0}$  \\
    \textbf{h(V\textsubscript{2})=-2}  & $26+0~\cancel{-}(\cancel{-}~2)$  & $0 ~\cancel{+(-2)}\cancel{-(-2)}$  & $1+(-4)~\cancel{-}(\cancel{-}~2)$  & $13+0~\cancel{-}(\cancel{-}~2)$  & $18+0~\cancel{-}(\cancel{-}~2)$  \\
    \textbf{h(V\textsubscript{3})=-4}  & $25+0~\cancel{-}(\cancel{-}~4)$  & $30+(-2)~\cancel{-}(\cancel{-}~4)$  & $0~\cancel{+(-4)}\cancel{-(-4)}$  & $30+0~\cancel{-}(\cancel{-}~4)$  & $17+0~\cancel{-}(\cancel{-}~4)$  \\
    \textbf{h(V\textsubscript{4})= 0}  & $26~\cancel{+0-0}$  & 0 +(-2)-0  & 1+(-4)-0  & $0~\cancel{+0-0}$  & $18~\cancel{+0-0}$  \\
    \textbf{h(V\textsubscript{5})= 0}  & $8~ \cancel{+0-0}$  & 13+(-2)-0  & 0+(-4)-0  & $13~\cancel{+0-0}$  & $0 ~\cancel{+0-0}$  \\ \bottomrule
    \end{tabular}
    \caption{Cálculos de ajuste de distancias reales}
    \label{AjusteJohnsonCalculos}
\end{table}

\begin{table}[H]
    \centering
    \begin{tabular}{@{}lccccc@{}}
    \toprule
    \textbf{origen} & \textbf{V\textsubscript{1}} & \textbf{V\textsubscript{2}} & \textbf{V\textsubscript{3}} & \textbf{V\textsubscript{4}} & \textbf{V\textsubscript{5}} \\ \midrule
    \textbf{V\textsubscript{1}}  &  0  & 12  & -3  & 27  & 18  \\
    \textbf{V\textsubscript{2}}  & 28  &  0  & -1  & 15  & 20  \\
    \textbf{V\textsubscript{3}}  & 29  & 32  &  0  & 34  & 21  \\
    \textbf{V\textsubscript{4}}  & 26  & -2  & -3  &  0  & 18  \\
    \textbf{V\textsubscript{5}}  &  8  & 11  & -4  & 13  &  0  \\ \bottomrule
    \end{tabular}
    \caption{Valores reales de distancias mínimas}
    \label{AjusteJohnsonValores}
\end{table}
    
\subsubsection{Reconstrucción caminos mínimos}\label{sec:parte1_4_4}

Para cualquier \texttt{(origen, destino)}, a partir del \cref{DijkstraGrafoHAntecesor}
podemos (recursivamenta desde \texttt{destino} al \texttt{origen})
reconstruir el camino mínimo, sabiendo que su peso es el calculado
en el \cref{AjusteJohnsonValores}. A continuación en la figura
\cref{GrafoH_minimos}, se muestran a modo de comprobación los
caminos mínimos según el vértice de \texttt{origen}:

\begin{figure}[H]
    \centering
    \subcaptionbox
        {\label{GrafoH_minimos1}\textbf{Caminos mínmos desde V\textsubscript{1}}.}
        {\includegraphics[width=0.4\linewidth,angle=0,origin=c]{out/caminos/camino1.png}}
    \subcaptionbox
        {\label{GrafoH_minimos2}\textbf{Caminos mínmos desde V\textsubscript{2}}.}
        {\includegraphics[width=0.4\linewidth,angle=0,origin=c]{out/caminos/camino2.png}}
    \subcaptionbox
        {\label{GrafoH_minimos3}\textbf{Caminos mínmos desde V\textsubscript{3}}.}
        {\includegraphics[width=0.4\linewidth,angle=0,origin=c]{out/caminos/camino3.png}}
    \subcaptionbox
        {\label{GrafoH_minimos4}\textbf{Caminos mínmos desde V\textsubscript{4}}.}
        {\includegraphics[width=0.4\linewidth,angle=0,origin=c]{out/caminos/camino4.png}}
    \subcaptionbox
        {\label{GrafoH_minimos5}\textbf{Caminos mínmos desde V\textsubscript{5}}.}
        {\includegraphics[width=0.4\linewidth,angle=0,origin=c]{out/caminos/camino5.png}}
    \caption{\label{GrafoH_minimos}\textbf{Grafo $H$} con caminos mínimos (en verde, junto con la distancia) según el punto de origen.}
\end{figure}

\end{document}
