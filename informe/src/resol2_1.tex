\documentclass[../tp2_grupo404.tex]{subfiles}

\graphicspath{{\subfix{../out/}}}

\begin{document}

\subsubsection{Breve descripción}

\paragraph{\underline{Greedy}}

Metodología de resolución de problemas de optimización
(minimización o maximización) que consiste en dividir el problema en
subproblemas con una jerarquía entre ellos. Cada subproblema se va
resolviendo iterativamente mediante una elección heurística y habilita
nuevos subproblemas.
\par Para cada problema existen diferentes algoritmos greedy, pero
algunos no resultan en la solución óptima y no todos los problemas
pueden resolverse mediante un algoritmo greedy.
\par Para poder aplicar esta metodología, un problema debe contener
las siguientes propiedades:
\begin{itemize}
    \item \underline{Elección greedy:} Se selecciona una solución óptima local
    esperando que la misma nos acerque a la solución óptima global.
    Para solucionar un subproblema, analiza el conjunto de los elementos
    del problema en el estado en que llegaron al mismo y elige
    heurísticamente la “mejor solución” local. Un subproblema está
    condicionado por las elecciones de los anteriores problemas y
    condiciona a los subproblemas siguientes.

    \item \underline{Subestructura óptima:} Un problema contiene una subestructura
    óptima si la solución óptima global del mismo contiene en su
    interior las soluciones óptimas de sus subproblemas. La elección
    greedy iterativamente resolverá los subproblemas óptimamente y nos
    llevará a la solución óptima global.

\end{itemize}

\paragraph{\underline{División y conquista}}
Consiste en dividir el problema en
subproblemas de igual naturaleza y menor tamaño. Se los conquista
(resuelve) en forma recursiva (hasta un caso base) y se combina los
resultados en una solución general. Generalmente se puede aplicar a
problemas donde la solución por fuerza bruta ya tiene una complejidad
polinómica. Analizar su complejidad requiere resolver una relación de
recurrencia.

\paragraph{\underline{Programación dinámica}}
Metodología de resolución de problemas de
optimización (minimización o maximización) que consiste en dividir el
problema en subproblemas con una jerarquía entre ellos (de menor a
mayor tamaño). Cada subproblema puede ser utilizado ser reutilizado en
diferentes subproblemas mayores.
\par Para poder resolverse de forma óptima utilizando programación
dinámica, un problema debe cumplir las siguientes propiedades:
\begin{itemize}
    \item \underline{Subestructura óptima:} Un problema contiene una subestructura
    óptima si la solución óptima global del mismo contiene en su interior
    las soluciones óptimas de sus subproblemas.

    \item \underline{Subproblemas superpuestos:} Un problema contiene subproblemas
    superpuestos si en la resolución de sus subproblemas vuelven a
    aparecer subproblemas previamente calculados.
\end{itemize}
La programación dinámica emplea \textbf{Memorización}: Técnica que consiste
en almacenar los resultados de los subproblemas previamente calculados para
evitar repetir su resolución cuando vuelva a requerirse. De esa forma
reducen la cantidad total de subproblemas a calcular, consiguiendo
reducir significativamente la complejidad temporal de la solución.



\subsubsection{Ventajas y desventajas}

\paragraph{\underline{Similitudes}}
\begin{itemize}
    \item En todos ellos se divide un problema en subproblemas.
    \item Greedy/Programación dinámica: Subestructura óptima, los subproblemas
    tienen una jerarquía entre ellos.
\end{itemize}

\paragraph{\underline{Ventajas}}
\begin{description}
    \item[Greedy] \leavevmode \begin{itemize}
        \item Suelen ser rápidos.
        \item Fáciles de implementar.
    \end{itemize}
 
    \item[Programación dinámica] \leavevmode \begin{itemize}
        \item La técnica de memorización permite reducir la complejidad temporal.
    \end{itemize}
 
    \item[División y conquista] \leavevmode \begin{itemize}
        \item Permite la resolución de problemas complejos.
        \item Es fácil de medir su complejidad.
    \end{itemize}
 
\end{description}

\paragraph{\underline{Desventajas}}
\begin{description}
    \item[Greedy] \leavevmode \begin{itemize}
        \item Es difícil de demostrar que para cada instancia de un problema se llega a la solución
            óptima mediante un algoritmo greedy.
        \item No todos los problemas se pueden solucionar con este método.
    \end{itemize}
 
    \item[Programación dinámica] \leavevmode \begin{itemize}
        \item Se necesita mucha memoria para almacenar los resultados de los subproblemas.
        \item Es posible que haya soluciones almacenadas en memoria que no se vuelvan a utilizar.
    \end{itemize}
 
    \item[División y conquista] \leavevmode \begin{itemize}
        \item Al utilizarse soluciones de forma recursiva, se utiliza mucha memoria
        del stack y esto puede traer problemas en casos grandes.
    \end{itemize}
 
\end{description}

\paragraph{\underline{¿Podría elegir una técnica sobre las otras?}}\leavevmode

No podría elegir una técnica sobre la otra, siempre dependerá de las propiedades
del problema que tenga que resolver y los recursos disponibles.

% FIN DEL DOCUMENTO (SECCIÓN P2.1)
% NO BORRAR POR ACCIDENTE NI ESCRIBIR COSSA ABAJO
\end{document}
