\documentclass[../tp2_grupo404.tex]{subfiles}

\graphicspath{{\subfix{../out/}}}

\begin{document}

El algoritmo de Johnson determina los caminos minimales entre todos
los pares de vértices de un grafo dirigido. Es preferible aplicarlo
sobre grafos de escasas aristas, ya que los procesa con mayor
rapidez que otros algoritmos de símil uso.

Su funcionamiento se basa en reponderaciones del grafo. Además,
utiliza los algoritmos de Dijkstra y Bellman-Ford como subrutinas.

La reponderación se ejecuta condicionalmente. Si todos los pesos
de las aristas son positivos, no es necesaria una reponderación,
basta con ejecutar Dijkstra una vez por vértice y devolviendo el
resultado final. En cambio, si el grafo (G) tiene aristas de peso
negativo, pero no ciclos negativos, computamos un nuevo set de
aristas con pesos no negativos que nos permita utilizar Dijkstra,
como en el caso anterior. A este proceso de cómputo de nuevas
aristas se lo llama reponderación, o cambio de peso.

El set de aristas obtenido de la reponderación debe cumplir dos
propiedades:
\begin{enumerate}
    \item[I] Se preservan los caminos más cortos.
    \item[II] No hay aristas de peso negativo.
\end{enumerate}

El proceso de reponderación tiene un costo de $O(V\times E)$.

Dado que la reponderación asegura que no quedan aristas de peso
negativo, se procesa por Dijkstra como en el caso anterior.
El último caso se da cuando el grafo tiene un ciclo negativo.

En instancias de este tipo, el algoritmo reporta que el grafo
contiene dicho ciclo y termina.

Optimalidad del algoritmo de Johnson

Supongamos una solución J para la instancia I, la cuál consta de un grafo del cual deseamos saber todos los
pesos de los caminos minimales entre todos sus vértices. J es una solución obtenida mediante el algoritmo de
Johnson. J será óptima si su matriz resultante, la obtenida al finalizar el algoritmo de Johnson, contiene a
todos los pesos de los caminos minimales del grafo de I.

Para hallar los caminos minimales, el algoritmo de Johnson, si decide que puede hacerse, aplica el algoritmo
de Djikstra sobre cada vértice del grafo y registra el resultado de cada uno de sus usos en una matriz. Dado
que el algoritmo de Djikstra asegura la obtención óptima de caminos minimales,
\footnote{Cormen tercera edición, pg 659-660}
el algoritmo de Johnson entonces guarda los pesos de dichos caminos minimales.

En los casos en los que Djikstra no es aplicable, y el grafo de I no contiene ciclos negativos, Johnson
aplica un reponderamiento sobre las aristas, brevemente explicado anteriormente, el cual permite la aplicación
de Djikstra. Esto ya fue probado que resulta en que el algoritmo de Johnson guardará los pesos de los caminos
minimales.

Por último, el caso del grafo de I con ciclo negativo, es redundante para nuestro problema, ya que simplemente
ejecutaríamos un loop infinito sobre dicho ciclo para obtener ganancias infinitas. Tarde o temprano esto se
vuelve inviable en una situación real. En este caso, el algoritmo de Johnson devuelve un mensaje alertando
sobre el ciclo negativo, ya que no es aplicable sobre un grafo de estas características.

En conclusión, el algoritmo de Johnson es óptimo, ya que retorna el resultado esperado, gracias a el uso de
otros algoritmos óptimos a su vez.

% FIN DEL DOCUMENTO (SECCIÓN P1.1)
% NO BORRAR POR ACCIDENTE NI ESCRIBIR COSSA ABAJO
\end{document}
