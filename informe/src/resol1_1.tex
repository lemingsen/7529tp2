\documentclass[../tp2_grupo404.tex]{subfiles}

\graphicspath{{\subfix{../out/}}}

\begin{document}

El algoritmo de Johnson determina los caminos minimales entre todos
los pares de vértices de un grafo dirigido. Es preferible aplicarlo
sobre grafos de escasas aristas, ya que los procesa con mayor
rapidez que otros algoritmos de símil uso.

Su funcionamiento se basa en reponderaciones del grafo. Además,
utiliza los algoritmos de Djikstra y Bellman-Ford como subrutinas.

La reponderación se ejecuta condicionalmente. Si todos los pesos
de las aristas son positivos, no es necesaria una reponderación,
basta con ejecutar Djikstra una vez por vértice y devolviendo el
resultado final. En cambio, si el grafo (G) tiene aristas de peso
negativo, pero no ciclos negativos, computamos un nuevo set de
aristas con pesos no negativos que nos permita utilizar Djikstra,
como en el caso anterior. A este proceso de cómputo de nuevas
aristas se lo llama reponderación, o cambio de peso.

El set de aristas obtenido de la reponderación debe cumplir dos
propiedades:
\begin{enumerate}
    \item[I] Se preservan los caminos más cortos.
    \item[II] No hay aristas de peso negativo.
\end{enumerate}

El proceso de reponderación tiene un costo de $O(V\times E)$.

Dado que la reponderación asegura que no quedan aristas de peso
negativo, se procesa por Djikstra como en el caso anterior.
El último caso se da cuando el grafo tiene un ciclo negativo.

En instancias de este tipo, el algoritmo reporta que el grafo
contiene dicho ciclo y termina.

% FIN DEL DOCUMENTO (SECCIÓN P1.1)
% NO BORRAR POR ACCIDENTE NI ESCRIBIR COSSA ABAJO
\end{document}
