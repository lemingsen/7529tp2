\documentclass[../tp2_grupo404.tex]{subfiles}

\graphicspath{{\subfix{../out/}}}

\begin{document}

Depende. Debe considerarse qué opción es la más eficiente, no sólo
en su tiempo de ejecución, sino también con respecto a los recursos
disponibles. Si, PD es más eficiente que G, el \emph{conundrum} está
en si los recursos son suficientes para hacer correr el algoritmo PD.
Entonces, el algoritmo a ser ejecutado depende de la cantidad de
elementos que posea la instancia a solucionar.

Si la instancia es pequeña, es decir aplicable PD en tanto a los
recursos computacionales disponibles, se utilizará PD. En caso
contrario, la instancia es tan grande que deberá utilizarse G, dado
que no es posible ejecutar PD.

En conclusión, conviene tener programados ambos algoritmos, ya que
dependiendo de la situación previamente explicada, y otros posibles
factores como preferencias institucionales o de algún usuario demasiado
insistente, convendrá utilizar o uno u otro algoritmo.

% FIN DEL DOCUMENTO (SECCIÓN P2.2)
% NO BORRAR POR ACCIDENTE NI ESCRIBIR COSSA ABAJO
\end{document}
