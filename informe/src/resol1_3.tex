\documentclass[../tp2_grupo404.tex]{subfiles}

\graphicspath{{\subfix{../out/}}}

\begin{document}

El algoritmo de Johnson puede ser más rápido que el de Floyd-Warshall en el caso  
de grafos dispersos (pocas aristas), pero el de Floyd-Warshall es más rápido  
cuando el grafo es denso (muchas aristas). La razón por la cual el algoritmo de  
Johnson es mejor para grafos dispersos es debido a que su complejidad temporal  
depende de la cantidad de aristas en el grafo, mientras que el de Floyd-Warshall no. 
\par 
El algoritmo de Johnson tiene una complejidad temporal de $O(V^2.log(V)+|V|.|E|)$.  
Por ende, si la cantidad de aristas es pequeña, correrá más rápido que con el  
tiempo $O(V^3)$ del de Floyd-Warshall.

% FIN DEL DOCUMENTO (SECCIÓN P1.3)
% NO BORRAR POR ACCIDENTE NI ESCRIBIR COSSA ABAJO
\end{document}
