\documentclass[../tp2_grupo404.tex]{subfiles}

\graphicspath{{\subfix{../out/}}}

\begin{document}

Para grafos densos, el algoritmo de Johnson es mejor ya que su complejidad
temporal depende de la cantidad de aristas.

Entonces, cuando el grafo es disperso el tiempo total del algoritmo
puede ser menor que el algoritmo de Floyd-Warshall, que resuelve el
mismo problema en un tiempo de $O(V^3)$.
\footnote{Wikipedia (autores varios).
\href{https://es.wikipedia.org/wiki/Algoritmo_de_Johnson}{\guillemotleft Algoritmo de Johnson \guillemotright}}
Además, la complejidad temporal de Johnson es mejor que
repetir Bellman-Ford por cada vértice, porque Johnson realiza una sola
iteración de Bellman-Ford para elimina los pesos negativos,
y luego aplica Dijkstra que tiene una complejidad menor.

% FIN DEL DOCUMENTO (SECCIÓN P1.3)
% NO BORRAR POR ACCIDENTE NI ESCRIBIR COSSA ABAJO
\end{document}
