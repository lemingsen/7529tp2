\documentclass[../tp2_grupo404.tex]{subfiles}

\graphicspath{{\subfix{../out/}}}

\begin{document}

Sí, el algoritmo de Johnson emplea la metodología de programación
dinámica, ya que utiliza el algoritmo de Bellman-Ford, que es un
algoritmo de programación dinámica, para computar una transformación
al grafo inicial para eliminar los pesos negativos.
Se almacenan resultados parciales de menor jerarquía
(distancia mínima estimada) que serán empleados como parte del cálculo en iteraciones posteriores.

% FIN DEL DOCUMENTO (SECCIÓN P1.6)
% NO BORRAR POR ACCIDENTE NI ESCRIBIR COSSA ABAJO
\end{document}
