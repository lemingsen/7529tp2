\documentclass[../tp2_grupo404.tex]{subfiles}

\graphicspath{{\subfix{../out/}}}

\begin{document}

Sí, el algoritmo de Johnson emplea la metodología de programación
dinámica, ya que tanto Bellman-Ford como Dijkstra, que son empleados
internamente, lo hacen. En ambos casos se almacenan resultados
parciales de menor jerarquía (distancia mínima estimada) que serán
empleados como parte del cálculo en iteraciones posteriores.

% FIN DEL DOCUMENTO (SECCIÓN P1.6)
% NO BORRAR POR ACCIDENTE NI ESCRIBIR COSSA ABAJO
\end{document}
