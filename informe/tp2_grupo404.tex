\documentclass[titlepage,a4paper]{article}
\PassOptionsToPackage{table,xcdraw}{xcolor}

\usepackage{a4wide}
\usepackage[colorlinks=true,linkcolor=black,urlcolor=blue,bookmarksopen=true]{hyperref}
\usepackage[utf8]{inputenc}
\usepackage[T1]{fontenc}
\usepackage{amssymb}
\usepackage{bookmark}
\usepackage{booktabs}
\usepackage{cancel}
\usepackage{caption}
\usepackage{csquotes}
\usepackage{subcaption}
\usepackage{fancyhdr}
\usepackage{fancyvrb,newverbs,xcolor} % Para lcverbatim
\usepackage[spanish]{babel}
\usepackage{graphicx}
\usepackage{float}
\usepackage[most]{tcolorbox}
\usepackage{longtable}
\usepackage{multirow}
\usepackage{standalone}
\usepackage{subfiles}
\usepackage{hyperref}
\usepackage{cleveref}

\crefname{table}{cuadro}{cuadros}
\crefname{figure}{figura}{figuras}
\crefname{section}{sección}{secciónes}
\crefname{subsection}{sección}{secciónes}
\crefname{subsubsection}{sección}{secciónes}

\pagestyle{fancy} % Encabezado y pie de página
\fancyhf{}
\fancyhead[L]{Grupo 404 - TP2}
\fancyhead[R]{Teoría de Algoritmos I - FIUBA}
\renewcommand{\headrulewidth}{0.4pt}
\fancyfoot[C]{\thepage}
\renewcommand{\footrulewidth}{0.4pt}

% lcverbatim = Cuadro de código con fondo gris
\definecolor{cverbbg}{gray}{0.93}
\newenvironment{lcverbatim}
 {\SaveVerbatim{cverb}}
 {\endSaveVerbatim
  \flushleft\fboxrule=0pt\fboxsep=.5em
  \colorbox{cverbbg}{%
    \makebox[\dimexpr\linewidth-2\fboxsep][l]{\BUseVerbatim{cverb}}%
  }
  \endflushleft
}

\begin{document}
\begin{titlepage} % Carátula
	\hfill\includegraphics[width=6cm]{img/logofiuba.jpg}
    \centering
    \vfill
    \Huge \textbf{Trabajo Práctico 2}
    \vskip2cm
    \Large [7529/9506] Teoría de Algoritmos I\\
    Segundo cuatrimestre de 2021
    \vfill
    \begin{tabular}{ | l | l | } % Datos del alumno
      \hline
      Grupo: & 404 \\ \hline
      Repositorio: & github.com/lucashemmingsen/7529tp2 \\ \hline
      Entrega: & nº 1 (20/10/2021) \\ \hline
  	\end{tabular}
    \vfill
    \includestandalone[mode=tex,width=\textwidth]{src/integrantes}
    \vfill
\end{titlepage}

\tableofcontents % Índice general
\newpage

\section{Introducción}\label{sec:intro}
\subsection{Resumen}
El presente informe documenta el enunciado y la solución del segundo trabajo práctico
de la materia Teoría de Algoritmos I. El mismo comprende el análisis del problema
planteado, la comparación de posibles algoritmos y la resolución manual del algoritmo
de Johnsosn; así como también respuestas a preguntas teóricas.

\subsection{Lineamientos básicos}
\begin{itemize}
    \item El trabajo se realizará en grupos de cinco personas.
    \item Se debe entregar el informe en formato pdf y código fuente en (.zip) en el aula virtual de la materia.
    \item El lenguaje de implementación es libre. Recomendamos utilizar C, C++ o Python. Sin embargo si se desea utilizar algún otro, se debe pactar con los docentes.
    \item Incluir en el informe los requisitos y procedimientos para su compilación y ejecución. La ausencia de esta información no permite probar el trabajo y deberá ser re-entregado con esta información.
    \item El informe debe presentar carátula con el nombre del grupo, datos de los integrantes y y fecha de entrega. Debe incluir número de hoja en cada página.
    \item En caso de re-entrega, entregar un apartado con las correcciones mencionadas
\end{itemize}

\newpage\section{Parte 1: Minimizando costos}\label{sec:parte1}

\textbf{Enunciado}

Una empresa productora de tecnología está planeando construir una fábrica para un
producto nuevo. Un aspecto clave en esa decisión corresponde a determinar dónde la
ubicarán para minimizar los gastos de logística y distribución. Cuenta con N depósitos
distribuidos en diferentes ciudades. En alguna de estas ciudades es donde deberá
instalar la nueva fábrica. Para los transportes utilizarán las rutas semanales con las
que ya cuentan. Cada ruta une dos depósitos en un sentido. No todos los depósitos
tienen rutas que los conecten. Por otro lado, los costos de utilizar una ruta tienen
diferentes valores. Por ejemplo hay rutas que requieren contratar más personal o
comprar nuevos vehículos. En otros casos son rutas subvencionadas y utilizarlas les
da una ganancia a la empresa. Otros factores que influyen son gastos de combustibles
y peajes. Para simplificar se ha desarrollado una tabla donde se indica para cada
ruta existente el costo de utilizarla (valor negativo si da ganancia).

Los han contratado para resolver este problema.

Han averiguado que se puede resolver el problema utilizando Bellman-Ford para cada
par de nodos o Floyd-Warshall en forma general. Un amigo les sugiere utilizar el
algoritmo de Johnson. Aclaración: ¡No existen ciclos negativos!

\noindent Se pide:

\begin{enumerate}
    \item Investigar el algoritmo de Johnson y explicar cómo funciona. ¿Es óptimo?
    \item En una tabla comparar la complejidad temporal y espacial de las tres propuestas.
    \item Analizar en qué situaciones una solución es mejor que otras.
    \item Crear un ejemplo con 5 depósitos y mostrar paso a paso cómo lo resolvería el algoritmo de Johnson.
    \item ¿Puede decirse que Johnson utiliza en su funcionamiento una metodología greedy? Justifique.
    \item ¿Puede decirse que Johnson utiliza en su funcionamiento una metodología de programación dinámica? Justifique.
    \item Programar la solución usando el algoritmo de Johnson.
\end{enumerate}

\subsubsection{Formato de los archivos}
Formato de los archivos:
El programa debe recibir por parámetro el path del archivo donde se encuentran los
costos entre cada depósito. El archivo debe ser de tipo texto y presentar por renglón,
separados por coma un par de depósitos con su distancia.

Ejemplo: “depositos.txt”

\begin{lcverbatim}
    A,B,54
    A,D,-3
    B,C,8
    …
\end{lcverbatim}

Debe resolver el problema y retornar por pantalla la solución. Debe mostrar por
consola en en que ciudad colocar el depósito. Además imprimir en forma de matriz los
costos mínimos entre cada uno de los depósitos.


\newpage\subsection{Algoritmo de Johnson: funcionamiento}\label{sec:parte1_1}
\begin{tcolorbox}[colback=blue!5!white,colframe=blue!75!black,title=Enunciado 1.1]
    Investigar el algoritmo de Johnson y explicar cómo funciona. ¿Es óptimo?
\end{tcolorbox}

\newpage\subsection{Comparación de algoritmos: Complejidad}\label{sec:parte1_2}
\begin{tcolorbox}[colback=blue!5!white,colframe=blue!75!black,title=Enunciado 1.2]
    En una tabla comparar la complejidad temporal y espacial de las tres propuestas.
\end{tcolorbox}

\newpage\subsection{Comparación de algoritmos: Ventajas y desventajas}\label{sec:parte1_3}
\begin{tcolorbox}[colback=blue!5!white,colframe=blue!75!black,title=Enunciado 1.3]
    Analizar en qué situaciones una solución es mejor que otras.
\end{tcolorbox}

\newpage\subsection{Algoritmo de Johnson: resolución manual}\label{sec:parte1_4}
\begin{tcolorbox}[colback=blue!5!white,colframe=blue!75!black,title=Enunciado 1.4]
    Crear un ejemplo con 5 depósitos y mostrar paso a paso cómo lo resolvería el algoritmo de Johnson.
\end{tcolorbox}

\subfile{src/resol1_4}

\newpage\subsection{Metodología Greedy}\label{sec:parte1_5}
\begin{tcolorbox}[colback=blue!5!white,colframe=blue!75!black,title=Enunciado 1.5]
    ¿Puede decirse que Johnson utiliza en su funcionamiento una metodología greedy? Justifique.
\end{tcolorbox}

\newpage\subsection{Programación dinámica}\label{sec:parte1_6}
\begin{tcolorbox}[colback=blue!5!white,colframe=blue!75!black,title=Enunciado 1.6]
    ¿Puede decirse que Johnson utiliza en su funcionamiento una metodología de programación dinámica? Justifique.
\end{tcolorbox}

\newpage\subsection{Programa con solución}\label{sec:parte1_7}
\begin{tcolorbox}[colback=blue!5!white,colframe=blue!75!black,title=Enunciado 1.7]
    Programar la solución usando el algoritmo de Johnson.
\end{tcolorbox}





\newpage\section{Parte 2: Un poco de teoría}\label{sec:parte2}

\subsection{Enunciado}

\begin{enumerate}
\item Hasta el momento hemos visto 3 formas distintas de resolver problemas.
    Greedy, división y conquista y programación dinámica. \begin{enumerate}
        \item Describa brevemente en qué consiste cada una de ellas.
        \item Identifique similitudes, diferencias, ventajas y
        desventajas entre las mismas. ¿Podría elegir una técnica sobre las otras?
    \end{enumerate}
\item Tenemos un problema que puede ser resuelto por un algoritmo Greedy (G) y
    por un algoritmo de Programación Dinámica (PD). G consiste en realizar múltiples
    iteraciones sobre un mismo arreglo, mientras que PD utiliza la información del
    arreglo en diferentes subproblemas a la vez que requiere almacenar dicha
    información calculada en cada uno de ellos, reduciendo así su complejidad;
    de tal forma logra que O(PD) < O(G). Sabemos que tenemos limitaciones en nuestros
    recursos computacionales (CPU y principalmente memoria).\\
    ¿Qué algoritmo elegiría para resolver el problema?\par
    Pista: probablemente no haya una respuesta correcta para este problema, solo justificaciones correctas.
\end{enumerate}

-------------

\newpage\subsection{Fromas de resolución de problemas}\label{sec:parte2_1}
\begin{tcolorbox}[colback=blue!5!white,colframe=blue!75!black,title=Enunciado 2.1]
    Hasta el momento hemos visto 3 formas distintas de resolver problemas.
    Greedy, división y conquista y programación dinámica. \begin{enumerate}
        \item Describa brevemente en qué consiste cada una de ellas.
        \item Identifique similitudes, diferencias, ventajas y
        desventajas entre las mismas. ¿Podría elegir una técnica sobre las otras?
    \end{enumerate}
\end{tcolorbox}

\newpage\subsection{Caso teórico puntual}\label{sec:parte2_2}
\begin{tcolorbox}[colback=blue!5!white,colframe=blue!75!black,title=Enunciado 2.2]
    Tenemos un problema que puede ser resuelto por un algoritmo Greedy (G) y
    por un algoritmo de Programación Dinámica (PD). G consiste en realizar múltiples
    iteraciones sobre un mismo arreglo, mientras que PD utiliza la información del
    arreglo en diferentes subproblemas a la vez que requiere almacenar dicha
    información calculada en cada uno de ellos, reduciendo así su complejidad;
    de tal forma logra que O(PD) < O(G). Sabemos que tenemos limitaciones en nuestros
    recursos computacionales (CPU y principalmente memoria).\\
    ¿Qué algoritmo elegiría para resolver el problema?
\end{tcolorbox}


% FIN DEL DOCUMENTO
% NO BORRAR POR ACCIDENTE NI ESCRIBIR COSSA ABAJO
\end{document}
